\chapter{Технологическая часть}

\section{Ошибки компилирования}

ANTLR4 обрабатыват все ошибки, которые возникают на этапах лексического анализа и синтаксического анализа. При это все возникающие исключения перехватываются и выводятся текстом в консоль.


\section{Генерация ANTLR4}

По грамматике языка программирования Pascal были сгенерированы следующие файлы с классами языка Python.

\begin{enumerate}
	\item PascalLexer -- лексический анализатор.
	\item PascalParser -- синтаксический анализатор.
	\item PascalListener -- абстрактный класс слушателя, с пустыми методами.
\end{enumerate}


\section{Обход сгенированного AST}

Для обхода дерева был создан класс-наследник базового класса слушателя (PascalListener) для переопределения <<\texttt{enter}>> и <<\texttt{exit}>> методов, которые соответствуют праивлам исходной гнрамматики.


\section{Пример компиляции программы}

В качестве примера приводится программа на языке Pascal, которая занимается простыми вычислениями и выводит результат через возврат главной функции. Она представлена в листинге \ref{lst:example-pascal} и сгенерированный LLVM(IR) код представлен в листинге \ref{lst:example-llvm}.

\mylisting[pascal]{example-pascal.pas}
	{}{Пример программы на Pascal}{example-pascal}{}

\mylisting[llvm]{example-llvm.ll}
	{}{Пример вывода LLVM(IR)}{example-llvm}{}
