\maketableofcontents

\intro

В современном мире программирование является важнейшим двигателем научно-технического прогресса и инноваций в самых различных областях: от разработки программного обеспечения для мобильных устройств и веб-приложений до создания сложных систем искусственного интеллекта и работы с большими данными. Программирование лежит в основе практически всех технологических достижений, обеспечивая автоматизацию процессов, повышение производительности и создание новых возможностей для решения задач, которые ранее считались невозможными.

Компиляторы играют ключевую роль в этой экосистеме, так как они обеспечивают связь между языками программирования высокого уровня и аппаратными ресурсами компьютеров. Языки программирования высокого уровня, такие как C, Java, Python, позволяют программистам разрабатывать программы, используя более абстрактные понятия и конструкции, которые проще для восприятия человека. Эти языки предоставляют возможность работать с переменными, функциями, объектами, структурами данных и другими высокоуровневыми концепциями, не требуя от разработчика знаний о том, как именно компьютер обрабатывает данные на уровне процессора.

Однако, несмотря на удобство работы с высокоуровневыми языками программирования, компьютерные процессоры не могут непосредственно исполнять код, написанный на этих языках. Процессоры работают только с машинным кодом — набором низкоуровневых инструкций, каждая из которых соответствует конкретной операции, такой как арифметические вычисления, загрузка данных в память или управление потоком исполнения программы. Компилятор берет на себя задачу трансляции высокоуровневого исходного кода в машинный код, который понятен и исполним процессором.

Целью работы является разработка компилятора языка программирования Pascal на языке Python для целевой платформы LLVM(IR) сс использованием генератора синтаксических анализаторов ANTLR4. Для ее достижения необходимо выполнить следующие задачи.

\begin{enumerate}
  \item Проанализировать предметную область.
  \item Определить грамматику языка Pascal.
  \item Сгенерировать синтаксическое дерево с помощью ANTL4.
  \item Сгенерировать выходной код с помощью LLVM(IR).
\end{enumerate}
